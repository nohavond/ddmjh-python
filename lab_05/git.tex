\documentclass{beamer}
\usetheme{Boadilla}

% packages ----------------------------------------%
\usepackage{graphicx}
\usepackage{tikz}
%--------------------------------------------------%

\title{Programování v jazyku Python}
\subtitle{Úvod do Gitu}
\author{Ondřej Nohava}
\institute[DDMJH]{Dům dětí a mládeže Jindřichův Hradec}
\date{}

\begin{document}

\begin{frame}
\titlepage
\end{frame}

\begin{frame}[t]{Github}



\begin{itemize}
	\item Github je UI pro Git
	
	\item zadarmo, stačí založit účet
	
	\item spousta výhod: záloha kódu, možnost sdílení kódu, spolupráce,...
\end{itemize}

\begin{tikzpicture}[remember picture, overlay]
\node[] at (current page.40) 
{
    \includegraphics[scale=0.6]{../../../Pictures/github-mark.png}
};
\end{tikzpicture}


\end{frame}



\begin{frame}[t]{Založení repozitáře}

	Repozitář -- místo pro náš projekt
	
	
	\vspace{0.3cm}	
	Postup při vytváření projektu:
\begin{enumerate}

	\item vytvoříme repozitář na githubu
	
	\item otevřeme si vscode, tam kde chceme repozitář zkopírovat
	
	\item pomocí terminálu vytvoříme repozitář v pc (git clone)

\end{enumerate}	
\end{frame}

\begin{frame}[t]{Práce s repozitářem}

\begin{block}{Užitečné příkazy}

\begin{itemize}

	\item git clone URL-Repozitáře -- vytvoří repozitář u nás na PC
	
	\item git add \{jméno\_souboru\} -- přidá soubor k poslání na Git
	
	\item git commit -m"Zpráva k přidaným souborům" -- přidá zprávu k souborům
	
	\item git push -- nahraje soubory na Git


\end{itemize}
\end{block}

\pause 

Při přidání každého souboru je nutný git add, commit a push v tomhle pořadí!


\end{frame}


\end{document}