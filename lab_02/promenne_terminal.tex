\documentclass{beamer}
\usetheme{Boadilla}

% packages ----------------------------------------%


\usepackage[utf8]{inputenc}

%--------------------------------------------------%

\title{Programování v jazyku Python}
\subtitle{Proměnné, základní vstup a výstup}
\author{Ondřej Nohava}
\institute[DDMJH]{Dům dětí a mládeže Jindřichův Hradec}
\date{}

\begin{document}

\begin{frame}
\titlepage
\end{frame}

\begin{frame}[t]{Terminál}

\begin{itemize}
	\item terminál se používá na základní vstup a výstup
\end{itemize}

\vspace{0.3cm}
\begin{center}
\includegraphics[scale=0.15]{../../../Pictures/Screenshots/terminal.png} 
\end{center}

\end{frame}

\begin{frame}[t]{Terminál}
\begin{itemize}
	\item terminál se používá na základní vstup a výstup
	
	\item vypsat text, čísla,... lze funkcí \textsf{print()}
	
	\item přečíst vstup z terminálu lze \textsf{input()}
	
	\pause
	
	\item Do fce \textsf{input()} lze přímo napsat otázku -- \textsf{input("Kolik ti je let?")}
\end{itemize}
\end{frame}


\begin{frame}[t]{Zkratky v terminálu}

\begin{itemize}
	\item python3 \{jméno\_programu\} -- spuštění programu
	
	\item $CTRL + C$ -- ukončení programu 'silou'
	
	\item $CTRL + L$ -- vyčištění terminálu
\end{itemize}

\end{frame}



\begin{frame}[t]{Typy vstupu}

\begin{itemize}
	\item \textsf{int(input())} -- náš vstup je číslo
	
	\item \textsf{input()} -- náš vstup je text 
	
	\pause	
	
	\begin{block}{Upozornění}
	S textem nelze provádět matematické operace, například $2 + 2$, $3 \cdot 4$.
	\end{block}
\end{itemize}

\vspace{0.3cm}

\begin{example}[Sčítání a násobení]
Vytvořte dva programy. Oba si postupně načtou dvě čísla a vrátí jejich součet/součin.

\end{example}


\end{frame}





\end{document}